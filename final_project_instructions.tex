% Options for packages loaded elsewhere
\PassOptionsToPackage{unicode}{hyperref}
\PassOptionsToPackage{hyphens}{url}
\PassOptionsToPackage{dvipsnames,svgnames,x11names}{xcolor}
%
\documentclass[
  letterpaper,
  DIV=11,
  numbers=noendperiod]{scrartcl}

\usepackage{amsmath,amssymb}
\usepackage{iftex}
\ifPDFTeX
  \usepackage[T1]{fontenc}
  \usepackage[utf8]{inputenc}
  \usepackage{textcomp} % provide euro and other symbols
\else % if luatex or xetex
  \usepackage{unicode-math}
  \defaultfontfeatures{Scale=MatchLowercase}
  \defaultfontfeatures[\rmfamily]{Ligatures=TeX,Scale=1}
\fi
\usepackage{lmodern}
\ifPDFTeX\else  
    % xetex/luatex font selection
\fi
% Use upquote if available, for straight quotes in verbatim environments
\IfFileExists{upquote.sty}{\usepackage{upquote}}{}
\IfFileExists{microtype.sty}{% use microtype if available
  \usepackage[]{microtype}
  \UseMicrotypeSet[protrusion]{basicmath} % disable protrusion for tt fonts
}{}
\makeatletter
\@ifundefined{KOMAClassName}{% if non-KOMA class
  \IfFileExists{parskip.sty}{%
    \usepackage{parskip}
  }{% else
    \setlength{\parindent}{0pt}
    \setlength{\parskip}{6pt plus 2pt minus 1pt}}
}{% if KOMA class
  \KOMAoptions{parskip=half}}
\makeatother
\usepackage{xcolor}
\setlength{\emergencystretch}{3em} % prevent overfull lines
\setcounter{secnumdepth}{-\maxdimen} % remove section numbering
% Make \paragraph and \subparagraph free-standing
\makeatletter
\ifx\paragraph\undefined\else
  \let\oldparagraph\paragraph
  \renewcommand{\paragraph}{
    \@ifstar
      \xxxParagraphStar
      \xxxParagraphNoStar
  }
  \newcommand{\xxxParagraphStar}[1]{\oldparagraph*{#1}\mbox{}}
  \newcommand{\xxxParagraphNoStar}[1]{\oldparagraph{#1}\mbox{}}
\fi
\ifx\subparagraph\undefined\else
  \let\oldsubparagraph\subparagraph
  \renewcommand{\subparagraph}{
    \@ifstar
      \xxxSubParagraphStar
      \xxxSubParagraphNoStar
  }
  \newcommand{\xxxSubParagraphStar}[1]{\oldsubparagraph*{#1}\mbox{}}
  \newcommand{\xxxSubParagraphNoStar}[1]{\oldsubparagraph{#1}\mbox{}}
\fi
\makeatother


\providecommand{\tightlist}{%
  \setlength{\itemsep}{0pt}\setlength{\parskip}{0pt}}\usepackage{longtable,booktabs,array}
\usepackage{calc} % for calculating minipage widths
% Correct order of tables after \paragraph or \subparagraph
\usepackage{etoolbox}
\makeatletter
\patchcmd\longtable{\par}{\if@noskipsec\mbox{}\fi\par}{}{}
\makeatother
% Allow footnotes in longtable head/foot
\IfFileExists{footnotehyper.sty}{\usepackage{footnotehyper}}{\usepackage{footnote}}
\makesavenoteenv{longtable}
\usepackage{graphicx}
\makeatletter
\def\maxwidth{\ifdim\Gin@nat@width>\linewidth\linewidth\else\Gin@nat@width\fi}
\def\maxheight{\ifdim\Gin@nat@height>\textheight\textheight\else\Gin@nat@height\fi}
\makeatother
% Scale images if necessary, so that they will not overflow the page
% margins by default, and it is still possible to overwrite the defaults
% using explicit options in \includegraphics[width, height, ...]{}
\setkeys{Gin}{width=\maxwidth,height=\maxheight,keepaspectratio}
% Set default figure placement to htbp
\makeatletter
\def\fps@figure{htbp}
\makeatother

\KOMAoption{captions}{tableheading}
\makeatletter
\@ifpackageloaded{caption}{}{\usepackage{caption}}
\AtBeginDocument{%
\ifdefined\contentsname
  \renewcommand*\contentsname{Table of contents}
\else
  \newcommand\contentsname{Table of contents}
\fi
\ifdefined\listfigurename
  \renewcommand*\listfigurename{List of Figures}
\else
  \newcommand\listfigurename{List of Figures}
\fi
\ifdefined\listtablename
  \renewcommand*\listtablename{List of Tables}
\else
  \newcommand\listtablename{List of Tables}
\fi
\ifdefined\figurename
  \renewcommand*\figurename{Figure}
\else
  \newcommand\figurename{Figure}
\fi
\ifdefined\tablename
  \renewcommand*\tablename{Table}
\else
  \newcommand\tablename{Table}
\fi
}
\@ifpackageloaded{float}{}{\usepackage{float}}
\floatstyle{ruled}
\@ifundefined{c@chapter}{\newfloat{codelisting}{h}{lop}}{\newfloat{codelisting}{h}{lop}[chapter]}
\floatname{codelisting}{Listing}
\newcommand*\listoflistings{\listof{codelisting}{List of Listings}}
\makeatother
\makeatletter
\makeatother
\makeatletter
\@ifpackageloaded{caption}{}{\usepackage{caption}}
\@ifpackageloaded{subcaption}{}{\usepackage{subcaption}}
\makeatother

\ifLuaTeX
  \usepackage{selnolig}  % disable illegal ligatures
\fi
\usepackage{bookmark}

\IfFileExists{xurl.sty}{\usepackage{xurl}}{} % add URL line breaks if available
\urlstyle{same} % disable monospaced font for URLs
\hypersetup{
  pdftitle={30538 Final Project: Reproducible Research - Volunteerism, Engagement, and Polarization in the U.S.},
  pdfauthor={Andrew White, Charles Huang, Justine Silverstein},
  colorlinks=true,
  linkcolor={blue},
  filecolor={Maroon},
  citecolor={Blue},
  urlcolor={Blue},
  pdfcreator={LaTeX via pandoc}}


\title{30538 Final Project: Reproducible Research - Volunteerism,
Engagement, and Polarization in the U.S.}
\author{Andrew White, Charles Huang, Justine Silverstein}
\date{2024-12-07}

\begin{document}
\maketitle


\section{1. Background}\label{background}

This project began as a shared interest in trends behind volunteering
rates in America, as two of our members (Justine and Charles) are
AmeriCorps alumni.

For the past few years, concerns about the American public's increasing
rates of isolation, decreasing lack of civic engagement and faith in
institutions, and greater rates of political polarization have been
prominent in the news and media. Our personal experiences with
AmeriCorps and volunteering have taught us that volunteering can be
effective at reducing isolation, increasing civic engagement/community
awareness, and decreasing negative polarization towards ``the other
side''. However, is volunteering a legitimate part of a public policy
solution to these issues, or is it just a red herring?

Our research questions were: 1. What is the current state of
volunteerism, political engagement and polarization in America? 2. What
factors make people more likely to volunteer or be civically engaged?

\section{2. Data Importing/Cleaning}\label{data-importingcleaning}

Our datasets for this project were:

\begin{enumerate}
\def\labelenumi{\arabic{enumi}.}
\tightlist
\item
  AmeriCorps CEV (Civic Engagement and Volunteering Supplement) for 2021
\item
  U.S. Census Bureau Volunteering and Civic Life Supplement - September
  2021
\item
  ANES (American National Election Studies) Time Series Data, 2020
\end{enumerate}

\#1 and \#2 primarily contain respondent information about volunteering
and measures of civic engagement, while \#3 contains information on
political affiliation and polarization.

We are importing the data from the AmeriCorps and ANES websites. Because
the datasets are over 100 MB, we include a Google Drive link here:

https://drive.google.com/drive/folders/1PUTN2pyh78MLoK0RVtGnf1ZwiM1BAAuV?usp=sharing

\section{2a. Data cleaning - CEV/VCL
data}\label{a.-data-cleaning---cevvcl-data}

As there are over 400 variables in the CEV and VCL data, here are the
most relevant variables we focused on:

Frequency and Type of Volunteering: PES16: Did the respondent spend any
time volunteering for any organization in the past 12 months? PES16D:
Frequency of volunteering (e.g., basically every day, a few times a
week). PTS16E: Approximate hours spent volunteering.

Political Engagement: PES2: How often the respondent discussed
political, societal, or local issues with friends or family. PES5: How
often these discussions occurred with neighbors. PES13: Contact or
visits to a public official to express opinions. PES14: Boycotting or
buying products based on political values or business practices.

Civic Participation and Group Membership: PES15: Belonging to groups,
organizations, or associations in the past 12 months. Neighbor and
Community Interaction: PES7: Participation in activities to improve
their neighborhood or community. Voting Behavior: PES11: Whether the
respondent voted in the last local elections.

Social Media and News Consumption: PES9: Posting views about political,
societal, or local issues on the internet or social media. PES10:
Frequency of consuming news related to political or societal issues.

Basic Demographics: Age: PRTAGE (Person's age) Gender: PESEX (Sex of the
respondent) Race/Ethnicity: PTDTRACE (Detailed race and Hispanic origin)
Marital Status: PEMARITL (Marital status of the respondent) Household
Composition: HRNUMHOU (Number of persons in the household)

Potential Confounding Variables Income: HEFAMINC (Household family
income level) Education: PEEDUCA (Highest level of school completed)
Urban/Rural Status: GTMETSTA (Metropolitan or non-metropolitan status)
Community Involvement: PES7 (Participation in neighborhood or community
activities) Social Media Use: PES9 (Posting views about political,
societal, or local issues on the internet or social media)

\begin{longtable}[]{@{}llllllllllllllllllllll@{}}
\toprule\noalign{}
& hrhhid & hrmonth & hryear4 & hurespli & hufinal & hetenure & hehousut
& hetelhhd & hetelavl & hephoneo & ... & pes17 & pes18 & prsupvol &
pwnrwgt & peswp1a & peswp1b & peswp1c & peswp1d & peswp2 & pwsrwgt \\
\midrule\noalign{}
\endhead
\bottomrule\noalign{}
\endlastfoot
0 & 333550513043249 & 9 & 2021 & -1 & 226 & 2 & 1 & 1 & -1 & 1 & ... &
-1 & -1 & -1 & 0 & -1 & -1 & -1 & -1 & -1 & 0 \\
1 & 900419145210736 & 9 & 2021 & 2 & 225 & 2 & 1 & 1 & -1 & 1 & ... & -1
& -1 & -1 & 0 & -1 & -1 & -1 & -1 & -1 & 0 \\
2 & 206408571810641 & 9 & 2021 & 4 & 225 & 2 & 1 & 1 & -1 & 1 & ... & -1
& -1 & -1 & 0 & -1 & -1 & -1 & -1 & -1 & 0 \\
3 & 606110727510599 & 9 & 2021 & -1 & 225 & 1 & 1 & -1 & -1 & 1 & ... &
-1 & -1 & -1 & 0 & -1 & -1 & -1 & -1 & -1 & 0 \\
4 & 848450067381002 & 9 & 2021 & -1 & 216 & -1 & 1 & -1 & -1 & 0 & ... &
-1 & -1 & -1 & 0 & -1 & -1 & -1 & -1 & -1 & 0 \\
\end{longtable}

One data cleaning issue we encountered with the CEV/VCL data: the data
is a mix of numeric code and qualitative input. We can create a mapping
function to swap the numeric codes with qualitative input, but the
existing qualitative input is outside of the data dictionary, so it
won't get picked up by any mapping functions and will be transmuted into
NaN data. We made a function that identifies all the values in the data
that aren't picked up by our data dictionaries- this function is located
in our config.py file.

For example, if there are some entries in a column that are already
coded ``Yes'' or ``No'' in addition to ``-1'', ``1'', ``2'', etc. our
existing mapping won't account for them and will turn them into NANs. We
want to catch those and account for them.

\section{CEV/VCL Data - Measuring Political
Engagement}\label{cevvcl-data---measuring-political-engagement}

As part of our analysis, we gauged volunteerism by referring to the
question ``Did you volunteer in the last 12 months''? However, there
isn't a single ``civic/political engagement'' question in the CEV/VCL
data, but rather several different questions that are related. We chose
five of the most relevant questions and weighted each based on their
level of effort:

\begin{enumerate}
\def\labelenumi{\arabic{enumi}.}
\tightlist
\item
  ``How frequently do you talk to a family member/neighbor about
  politics?'' (15\%)
\item
  ``How frequently do you post political views on social media?'' (15\%)
\item
  ``How frequently do you consume political news/media?'' (10\%)
\item
  ``Did you contact an elected official to express your opinion in the
  last 12 months?'' (30\%)
\item
  ``Did you boycott a company based on their values in the last 12
  months?'' (30\%)
\end{enumerate}

This generated a score from 0 - 100 that we could use as a (imperfect)
proxy for political engagement. We mutated a new variable,
political\_engagement\_score, to measure this and added it to our
dataset.

(Important caveat: For the political engagement questions, less than
20\% of respondents answered three or more of the selected questions. To
ensure meaningful data, we excluded all respondents who did not meet
this threshold. While this approach improves the consistency of the
dataset, we should be aware of potential selection bias.)

\section{ANES data cleaning}\label{anes-data-cleaning}

For the ANES data, we create

\begin{verbatim}
Columns before adding engagement score:
['Household_ID', 'Household_ID_2', 'Volunteered_Past_Year', 'Volunteering_Frequency', 'Hours_Spent_Volunteering', 'Discussed_Issues_With_Friends_Family', 'Discussed_Issues_With_Neighbors', 'Contacted_Public_Official', 'Boycott_Based_On_Values', 'Belonged_To_Groups', 'Community_Improvement_Activities', 'Voted_In_Local_Election', 'Posted_Views_On_Social_Media', 'Frequency_Of_News_Consumption', 'Age', 'Gender', 'Race_Ethnicity', 'Marital_Status', 'Household_Size', 'US State', 'Family_Income_Level', 'Education_Level', 'Urban_Rural_Status']

Shape before: (255744, 23)

Columns after adding engagement score:
['Household_ID', 'Household_ID_2', 'Volunteered_Past_Year', 'Volunteering_Frequency', 'Hours_Spent_Volunteering', 'Discussed_Issues_With_Friends_Family', 'Discussed_Issues_With_Neighbors', 'Contacted_Public_Official', 'Boycott_Based_On_Values', 'Belonged_To_Groups', 'Community_Improvement_Activities', 'Voted_In_Local_Election', 'Posted_Views_On_Social_Media', 'Frequency_Of_News_Consumption', 'Age', 'Gender', 'Race_Ethnicity', 'Marital_Status', 'Household_Size', 'US State', 'Family_Income_Level', 'Education_Level', 'Urban_Rural_Status', 'political_engagement_score', 'engagement_level']

Shape after: (255744, 25)

Verifying engagement score columns exist:
'political_engagement_score' exists: True
'engagement_level' exists: True
\end{verbatim}

\begin{verbatim}

Columns in saved CSV:
['Household_ID', 'Household_ID_2', 'Volunteered_Past_Year', 'Volunteering_Frequency', 'Hours_Spent_Volunteering', 'Discussed_Issues_With_Friends_Family', 'Discussed_Issues_With_Neighbors', 'Contacted_Public_Official', 'Boycott_Based_On_Values', 'Belonged_To_Groups', 'Community_Improvement_Activities', 'Voted_In_Local_Election', 'Posted_Views_On_Social_Media', 'Frequency_Of_News_Consumption', 'Age', 'Gender', 'Race_Ethnicity', 'Marital_Status', 'Household_Size', 'US State', 'Family_Income_Level', 'Education_Level', 'Urban_Rural_Status', 'political_engagement_score', 'engagement_level']
Dataset saved to: shiny-app/basic-app/data/cev_2021_cleaned.csv
\end{verbatim}

\section{Data Cleaning - ANES Data}\label{data-cleaning---anes-data}

As with the CEV/VCL data, our goal was to subset the data so that it
only contains relevant variables. We accomplish this by making two
lists:

List 1: This list is designed to capture variables covering geographic
information (V201011, V201013a, V201013b, V201014a, V201014b)

List 2: This list is designed to capture variables covering information
about assessments of political positioning (i.e.~left, right, center)

\section{ANES - More Data Cleaning}\label{anes---more-data-cleaning}

Analyzing Question V201200, which is a question asking:

``Where would you place yourself on this scale, or haven't you thought
much about this? Value Labels-9. Refused -8. Don't know 1. Extremely
Liberal 2. Liberal 3. Slightly Liberal 4. Moderate; middle of the road
5. Slightly Conservative 6. Conservative 7. Extremely Conservative 99.
Haven't thought much about this''

We use this data to make a dataframe aggregated by state, and then we
can show correlation between measure of polarity and the share of
respondents in a state who did volunteer work.

Note that we previously used two ANES variables as US State variables,
with only ``US State 2'' being used in analysis, purely because it has
more in-universe entries. This appears to be partially due to respondent
reactions to different questions, and partially due to information
restrictions on the dataset. As such, ``US State'' is only used for CEV
data.

Then, make data using V201228, which asks:

``Generally speaking, do you usually think of yourself as {[}a Democrat,
a Republican / a Republican, a Democrat{]}, an independent, or what?''

-9. Refused -8. Don't know -4. Technical error 0. No preference \{VOL -
video/phone only\} 1. Democrat 2. Republican 3. Independent 5. Other
party \{SPECIFY\}

\section{2. Exploratory Analysis}\label{exploratory-analysis}

\section{2a. Measures of
Polarization}\label{a.-measures-of-polarization}

We will use two measures of polarization from the ANES data; each
provides some amount of information that can be interpreted to indicate
polarization to a certain extent, though both have their drawbacks.

\begin{enumerate}
\def\labelenumi{\arabic{enumi}.}
\tightlist
\item
  Share of Outliers - we create a series of functions that group
  respondents by party Democrats with conservative-leaning ideologies,
  and Republicans with liberal-leaning ones.
\end{enumerate}

\begin{longtable}[]{@{}lllll@{}}
\toprule\noalign{}
& Party\_Affiliation\_(V201228) & Outliers & Party\_Count &
Percent\_Outliers \\
\midrule\noalign{}
\endhead
\bottomrule\noalign{}
\endlastfoot
0 & Democrat & 0 & 137 & 0.0000 \\
1 & Independent & 0 & 0 & 0.0000 \\
2 & Other party & 0 & 0 & 0.0000 \\
3 & Refused & 0 & 0 & 0.0000 \\
4 & Republican & 4 & 84 & 0.0181 \\
\end{longtable}

In a paper on quantifying polarization written by Aaron Bramson et al
(https://inferenceproject.yale.edu/sites/default/files/688938.pdf), the
authors examine a range of polarization indicators. A relatively simple
(and in some ways problematic) measurement is called spread, or
dispersion- essentially the gap between the most extreme political
positions.

In the paper, Bramson et al.~explain: ``Polarization in the sense of
spread can be measured as the value of the agent with the highest belief
value minus the value of the agent with the lowest belief value
(sometimes called the `range' of the data).''\,''

We (imperfectly) approximate this using two more variables: V201206 and
V201207. These ask respondents to position political parties on the
political spectrum. We can select the most ideologically distant nodes
on the personal ideology scale (extremely liberal and extremely
conservative) and capture how far apart their conceptions of each party
are, on average, and then disaggregate by state.

We will assign the different ideological positions to different points
on a spectrum, namely: -3, -2, and -1 are ``Extremely Liberal'',
``Liberal'', and ``Slightly Liberal'';

and: 0 is ``Moderate; middle of the road'';

finally, 1, 2, and 3 are ``Slightly conservative'', ``Conservative'' and
``Extremely conservative.''\,''

We'll then compare average positions by state. For example, if the
average extremely liberal respondent in Texas places Democrats at at -1
(slightly liberal) and the average for the extremely conservative
respondents in -3 (extremely liberal), then the distance between the two
is 4, meaning Texas would have a spread of 4 for this question.

We create the crosswalks V201206 and V201207 -

We use the Positioning columns to compare the average party position
selections between the 2 extremes.

Step 1: First grouping

Step 2: We use the variable position\_groups to create a dataframe that
has 4 columns: (1) The position Liberals give Democrats on the spectrum
(2) the position Conservatives give Democrats on the spectrum (3) the
position Liberals give Republicans on the spectrum (4) the position
Conservatives give Republicans on the spectrum

Step 3: Now, we use those 4 columns to create the spread, meaning the
absolute value of the difference betweeen:

\begin{enumerate}
\def\labelenumi{(\arabic{enumi})}
\item
  The position Liberals give Democrats on the spectrum and the position
  Conservatives give Democrats on the spectrum
\item
  The position Liberals give Republicans on the spectrum and the
  position Conservatives give Republicans on the spectrum
\end{enumerate}

Step 4: Now, we graph those differences and interpret these spreads as
an indicator of polarization. We acknowledge that this is too simple an
analysis to account for the full complexity of this kind of measurement,
as polarization involves not just distance between extremes, but also
clustering around them. We also acknowledge that our political scale
assumes a linear ideological spectrum, which isn't always the case.

Step 5: We merge the two datasets together on ``US State 2''. Note that
this merged dataset could be misleading, because some of the merged
variables only apply to the ``extreme'' respondents in each state, and
so it's not representative of the entire state's respondents'
positioning of different parties. We've dropped those variables from the
merge to try to account for this.

Step 6: We merge the AmeriCorps CEV/VCL data with the spread data so far

Step 7: We create a function for viewing Political Engagement alongside
spread, one generating a table with 2 states for comparison, and another
2 returning graphs

Below is an example of using our function to generate a political
engagement score for different states:

\begin{verbatim}
   US State  political_engagement_score  Spread_Dem  Spread_Repub
13       IL                   27.038567    1.409091      0.196970
36       OR                   37.203540    2.454545      2.393939
\end{verbatim}

And below are sample static graphs on Democratic and Republican
positions:

\begin{verbatim}
alt.Chart(...)
\end{verbatim}

\begin{verbatim}
alt.Chart(...)
\end{verbatim}

\begin{verbatim}
alt.Chart(...)
\end{verbatim}

\begin{verbatim}
alt.Chart(...)
\end{verbatim}

Below is a atatic graph of ideological position in the US nationally as
of 2020:

\begin{verbatim}
ImportError: The "vegafusion" data transformer and chart.transformed_data feature requires
version 1.5.0 or greater of the 'vegafusion-python-embed' and 'vegafusion' packages.
These can be installed with pip using:
    pip install "vegafusion[embed]>=1.5.0"
Or with conda using:
    conda install -c conda-forge "vegafusion-python-embed>=1.5.0" "vegafusion>=1.5.0"

ImportError: No module named 'vegafusion'
---------------------------------------------------------------------------
ModuleNotFoundError                       Traceback (most recent call last)
File ~/Documents/GitHub/student30538/problem_sets/final_project/venv/lib/python3.13/site-packages/altair/utils/_importers.py:15, in import_vegafusion()
     14 try:
---> 15     import vegafusion as vf
     17     version = importlib_version("vegafusion")

ModuleNotFoundError: No module named 'vegafusion'

The above exception was the direct cause of the following exception:

ImportError                               Traceback (most recent call last)
File ~/Documents/GitHub/student30538/problem_sets/final_project/venv/lib/python3.13/site-packages/IPython/core/formatters.py:1036, in MimeBundleFormatter.__call__(self, obj, include, exclude)
   1033     method = get_real_method(obj, self.print_method)
   1035     if method is not None:
-> 1036         return method(include=include, exclude=exclude)
   1037     return None
   1038 else:

File ~/Documents/GitHub/student30538/problem_sets/final_project/venv/lib/python3.13/site-packages/altair/vegalite/v5/api.py:3682, in TopLevelMixin._repr_mimebundle_(self, *args, **kwds)
   3680 else:
   3681     if renderer := renderers.get():
-> 3682         return renderer(dct)

File ~/Documents/GitHub/student30538/problem_sets/final_project/venv/lib/python3.13/site-packages/altair/utils/display.py:232, in HTMLRenderer.__call__(self, spec, **metadata)
    230 kwargs = self.kwargs.copy()
    231 kwargs.update(**metadata, output_div=self.output_div)
--> 232 return spec_to_mimebundle(spec, format="html", **kwargs)

File ~/Documents/GitHub/student30538/problem_sets/final_project/venv/lib/python3.13/site-packages/altair/utils/mimebundle.py:129, in spec_to_mimebundle(spec, format, mode, vega_version, vegaembed_version, vegalite_version, embed_options, engine, **kwargs)
    127 internal_mode: Literal["vega-lite", "vega"] = mode
    128 if using_vegafusion():
--> 129     spec = compile_with_vegafusion(spec)
    130     internal_mode = "vega"
    132 # Default to the embed options set by alt.renderers.set_embed_options

File ~/Documents/GitHub/student30538/problem_sets/final_project/venv/lib/python3.13/site-packages/altair/utils/_vegafusion_data.py:258, in compile_with_vegafusion(vegalite_spec)
    255 # Local import to avoid circular ImportError
    256 from altair import data_transformers, vegalite_compilers
--> 258 vf = import_vegafusion()
    260 # Compile Vega-Lite spec to Vega
    261 compiler = vegalite_compilers.get()

File ~/Documents/GitHub/student30538/problem_sets/final_project/venv/lib/python3.13/site-packages/altair/utils/_importers.py:44, in import_vegafusion()
     33 except ImportError as err:
     34     msg = (
     35         'The "vegafusion" data transformer and chart.transformed_data feature requires\n'
     36         f"version {min_version} or greater of the 'vegafusion-python-embed' and 'vegafusion' packages.\n"
   (...)
     42         f"ImportError: {err.args[0]}"
     43     )
---> 44     raise ImportError(msg) from err

ImportError: The "vegafusion" data transformer and chart.transformed_data feature requires
version 1.5.0 or greater of the 'vegafusion-python-embed' and 'vegafusion' packages.
These can be installed with pip using:
    pip install "vegafusion[embed]>=1.5.0"
Or with conda using:
    conda install -c conda-forge "vegafusion-python-embed>=1.5.0" "vegafusion>=1.5.0"

ImportError: No module named 'vegafusion'
\end{verbatim}

\begin{verbatim}
alt.Chart(...)
\end{verbatim}

Note: We removed non-responses, likely creating a bias in the relative
size of the remaining groups (it's difficult to predict the direction of
said bias, however.)

Additionally, below is a sample static graph of ideological position by
state, using IL as an example:

\begin{verbatim}
ImportError: The "vegafusion" data transformer and chart.transformed_data feature requires
version 1.5.0 or greater of the 'vegafusion-python-embed' and 'vegafusion' packages.
These can be installed with pip using:
    pip install "vegafusion[embed]>=1.5.0"
Or with conda using:
    conda install -c conda-forge "vegafusion-python-embed>=1.5.0" "vegafusion>=1.5.0"

ImportError: No module named 'vegafusion'
---------------------------------------------------------------------------
ModuleNotFoundError                       Traceback (most recent call last)
File ~/Documents/GitHub/student30538/problem_sets/final_project/venv/lib/python3.13/site-packages/altair/utils/_importers.py:15, in import_vegafusion()
     14 try:
---> 15     import vegafusion as vf
     17     version = importlib_version("vegafusion")

ModuleNotFoundError: No module named 'vegafusion'

The above exception was the direct cause of the following exception:

ImportError                               Traceback (most recent call last)
File ~/Documents/GitHub/student30538/problem_sets/final_project/venv/lib/python3.13/site-packages/IPython/core/formatters.py:1036, in MimeBundleFormatter.__call__(self, obj, include, exclude)
   1033     method = get_real_method(obj, self.print_method)
   1035     if method is not None:
-> 1036         return method(include=include, exclude=exclude)
   1037     return None
   1038 else:

File ~/Documents/GitHub/student30538/problem_sets/final_project/venv/lib/python3.13/site-packages/altair/vegalite/v5/api.py:3682, in TopLevelMixin._repr_mimebundle_(self, *args, **kwds)
   3680 else:
   3681     if renderer := renderers.get():
-> 3682         return renderer(dct)

File ~/Documents/GitHub/student30538/problem_sets/final_project/venv/lib/python3.13/site-packages/altair/utils/display.py:232, in HTMLRenderer.__call__(self, spec, **metadata)
    230 kwargs = self.kwargs.copy()
    231 kwargs.update(**metadata, output_div=self.output_div)
--> 232 return spec_to_mimebundle(spec, format="html", **kwargs)

File ~/Documents/GitHub/student30538/problem_sets/final_project/venv/lib/python3.13/site-packages/altair/utils/mimebundle.py:129, in spec_to_mimebundle(spec, format, mode, vega_version, vegaembed_version, vegalite_version, embed_options, engine, **kwargs)
    127 internal_mode: Literal["vega-lite", "vega"] = mode
    128 if using_vegafusion():
--> 129     spec = compile_with_vegafusion(spec)
    130     internal_mode = "vega"
    132 # Default to the embed options set by alt.renderers.set_embed_options

File ~/Documents/GitHub/student30538/problem_sets/final_project/venv/lib/python3.13/site-packages/altair/utils/_vegafusion_data.py:258, in compile_with_vegafusion(vegalite_spec)
    255 # Local import to avoid circular ImportError
    256 from altair import data_transformers, vegalite_compilers
--> 258 vf = import_vegafusion()
    260 # Compile Vega-Lite spec to Vega
    261 compiler = vegalite_compilers.get()

File ~/Documents/GitHub/student30538/problem_sets/final_project/venv/lib/python3.13/site-packages/altair/utils/_importers.py:44, in import_vegafusion()
     33 except ImportError as err:
     34     msg = (
     35         'The "vegafusion" data transformer and chart.transformed_data feature requires\n'
     36         f"version {min_version} or greater of the 'vegafusion-python-embed' and 'vegafusion' packages.\n"
   (...)
     42         f"ImportError: {err.args[0]}"
     43     )
---> 44     raise ImportError(msg) from err

ImportError: The "vegafusion" data transformer and chart.transformed_data feature requires
version 1.5.0 or greater of the 'vegafusion-python-embed' and 'vegafusion' packages.
These can be installed with pip using:
    pip install "vegafusion[embed]>=1.5.0"
Or with conda using:
    conda install -c conda-forge "vegafusion-python-embed>=1.5.0" "vegafusion>=1.5.0"

ImportError: No module named 'vegafusion'
\end{verbatim}

\begin{verbatim}
alt.Chart(...)
\end{verbatim}

Lastly, we will use the merged AmeriCorps and ANES data to plot a
correlation between volunteer hours and political engagement:

\begin{verbatim}
ImportError: The "vegafusion" data transformer and chart.transformed_data feature requires
version 1.5.0 or greater of the 'vegafusion-python-embed' and 'vegafusion' packages.
These can be installed with pip using:
    pip install "vegafusion[embed]>=1.5.0"
Or with conda using:
    conda install -c conda-forge "vegafusion-python-embed>=1.5.0" "vegafusion>=1.5.0"

ImportError: No module named 'vegafusion'
---------------------------------------------------------------------------
ModuleNotFoundError                       Traceback (most recent call last)
File ~/Documents/GitHub/student30538/problem_sets/final_project/venv/lib/python3.13/site-packages/altair/utils/_importers.py:15, in import_vegafusion()
     14 try:
---> 15     import vegafusion as vf
     17     version = importlib_version("vegafusion")

ModuleNotFoundError: No module named 'vegafusion'

The above exception was the direct cause of the following exception:

ImportError                               Traceback (most recent call last)
File ~/Documents/GitHub/student30538/problem_sets/final_project/venv/lib/python3.13/site-packages/IPython/core/formatters.py:1036, in MimeBundleFormatter.__call__(self, obj, include, exclude)
   1033     method = get_real_method(obj, self.print_method)
   1035     if method is not None:
-> 1036         return method(include=include, exclude=exclude)
   1037     return None
   1038 else:

File ~/Documents/GitHub/student30538/problem_sets/final_project/venv/lib/python3.13/site-packages/altair/vegalite/v5/api.py:3682, in TopLevelMixin._repr_mimebundle_(self, *args, **kwds)
   3680 else:
   3681     if renderer := renderers.get():
-> 3682         return renderer(dct)

File ~/Documents/GitHub/student30538/problem_sets/final_project/venv/lib/python3.13/site-packages/altair/utils/display.py:232, in HTMLRenderer.__call__(self, spec, **metadata)
    230 kwargs = self.kwargs.copy()
    231 kwargs.update(**metadata, output_div=self.output_div)
--> 232 return spec_to_mimebundle(spec, format="html", **kwargs)

File ~/Documents/GitHub/student30538/problem_sets/final_project/venv/lib/python3.13/site-packages/altair/utils/mimebundle.py:129, in spec_to_mimebundle(spec, format, mode, vega_version, vegaembed_version, vegalite_version, embed_options, engine, **kwargs)
    127 internal_mode: Literal["vega-lite", "vega"] = mode
    128 if using_vegafusion():
--> 129     spec = compile_with_vegafusion(spec)
    130     internal_mode = "vega"
    132 # Default to the embed options set by alt.renderers.set_embed_options

File ~/Documents/GitHub/student30538/problem_sets/final_project/venv/lib/python3.13/site-packages/altair/utils/_vegafusion_data.py:258, in compile_with_vegafusion(vegalite_spec)
    255 # Local import to avoid circular ImportError
    256 from altair import data_transformers, vegalite_compilers
--> 258 vf = import_vegafusion()
    260 # Compile Vega-Lite spec to Vega
    261 compiler = vegalite_compilers.get()

File ~/Documents/GitHub/student30538/problem_sets/final_project/venv/lib/python3.13/site-packages/altair/utils/_importers.py:44, in import_vegafusion()
     33 except ImportError as err:
     34     msg = (
     35         'The "vegafusion" data transformer and chart.transformed_data feature requires\n'
     36         f"version {min_version} or greater of the 'vegafusion-python-embed' and 'vegafusion' packages.\n"
   (...)
     42         f"ImportError: {err.args[0]}"
     43     )
---> 44     raise ImportError(msg) from err

ImportError: The "vegafusion" data transformer and chart.transformed_data feature requires
version 1.5.0 or greater of the 'vegafusion-python-embed' and 'vegafusion' packages.
These can be installed with pip using:
    pip install "vegafusion[embed]>=1.5.0"
Or with conda using:
    conda install -c conda-forge "vegafusion-python-embed>=1.5.0" "vegafusion>=1.5.0"

ImportError: No module named 'vegafusion'
\end{verbatim}

\begin{verbatim}
alt.Chart(...)
\end{verbatim}

This shows a noted positive correlation between volunteer hours and
political engagement. However, as we'll see, this doesn't necessarily
imply that increasing volunteering will lead to higher civic engagement.

\begin{enumerate}
\def\labelenumi{\arabic{enumi}.}
\setcounter{enumi}{3}
\tightlist
\item
  Reproductibility (10\%)

  \begin{itemize}
  \tightlist
  \item
    The project and files should be structured and documented so that
    the TAs can clone your repository and reproduce your results (see
    ``Final Repository'' below) by knitting your \texttt{.qmd} and, if
    needed, downloading the dataset(s) you use using the link provided
    in the \texttt{.qmd} comments
  \end{itemize}
\item
  Git (10\%)

  \begin{itemize}
  \tightlist
  \item
    You should submit your project as a Git repository.
  \item
    Create multiple branches as you work for different pieces of the
    analysis. Branches may correspond to work done by different partners
    or to different features if you are working alone.
  \item
    Your final repository should have one branch: \texttt{main}
  \item
    We reserve the right to check the git commit history to ensure that
    all members have contributed to the project.
  \end{itemize}
\item
  Extra credit: text processing (up to 10\%)

  \begin{itemize}
  \tightlist
  \item
    Introduce some form of text analysis using natural language
    processing methods discussed in class.
  \end{itemize}
\end{enumerate}

\subsection{Writeup (15\%)}\label{writeup-15}

\begin{itemize}
\tightlist
\item
  You will then spend \emph{no more than 3 pages} writing up your
  project.
\item
  The primary purpose of this writeup is to inform us of what we are
  reading before we look at your code.
\item
  You should describe your research question, then discuss the approach
  you took and the coding involved, including discussing any weaknesses
  or difficulties encountered.
\item
  Display your static plots, and briefly describe them and your Shiny
  app. Discuss the policy implications of your findings.
\item
  Finish with a discussion of directions for future work.
\item
  The top of your writeup should include the names of all group members,
  their respective sections, and Github user names.
\end{itemize}

\subsection{Presentation (15\%)}\label{presentation-15}

\begin{itemize}
\tightlist
\item
  On the day of the presentation, one of the group members will be
  \emph{randomly selected} to give a \emph{8-minute in-class
  presentation}. All group members must be present.
\item
  Any group member who is not present will receive an automatic 0 for
  the presentation portion of the final project.
\item
  The presentation will be of slides that largely mirror the structure
  of the writeup, but will be more focused on discussing the research
  question and results as opposed to explaining the details of the
  coding.
\end{itemize}

\section{Final Repository}\label{final-repository}

Your final repository must contain the following:

\begin{itemize}
\tightlist
\item
  Documentation and Meta-data

  \begin{itemize}
  \tightlist
  \item
    A \texttt{requirements.txt} file
  \item
    A \texttt{.gitignore} file that ignores unneeded files
    (e.g.~\texttt{venv})
  \end{itemize}
\item
  Writeup: a user should be able to knit your \texttt{.qmd} file and
  re-generate the HTML version of your writeup

  \begin{itemize}
  \tightlist
  \item
    The \texttt{.qmd} file associated with your write-up
  \item
    An HTML and PDF'd version of your writeup
  \item
    A folder named \texttt{pictures} that contains the files for any
    pictures required to knit your writeup
  \end{itemize}
\item
  Data

  \begin{itemize}
  \tightlist
  \item
    A folder named \texttt{data} that contains the initial, unmodified
    dataframes you download and the final versions of the dataframe(s)
    you built.
  \item
    If the dataset is greater than 100MB, it can hosted on Drive or
    Dropbox and the link should be provided in your .\texttt{qmd} file
    as a comment
  \end{itemize}
\item
  Shiny app

  \begin{itemize}
  \tightlist
  \item
    A folder named \texttt{shiny-app} that contains the code and any
    additional files needed to deploy your app
  \item
    A user should be able to deploy your app directly from the command
    line within this folder
  \end{itemize}
\end{itemize}

\section{Key Dates}\label{key-dates}

\begin{itemize}
\tightlist
\item
  By November 1

  \begin{itemize}
  \tightlist
  \item
    Proposal submitted to Canvas quiz
  \item
    (Optional) meeting with Professor Ganong, Professor Shi, or Head TA
    Ozzy Houck
  \item
    Sign up for presentation slot
  \end{itemize}
\item
  December 2- December 5: in-class presentations
\item
  December 7, 5PM: final repository submitted via Gradescope
\end{itemize}




\end{document}
